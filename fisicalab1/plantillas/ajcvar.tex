% Qué tipo de documento estamos por comenzar:
\documentclass[11pt, a4paper]{article}
% Esto es para que el LaTeX sepa que el texto está en español:
\usepackage[spanish]{babel}
\selectlanguage{spanish}
% Esto es para poder escribir acentos directamente:
\usepackage[utf8]{inputenc}
\usepackage[T1]{fontenc}
\usepackage{float}
\addto\captionsspanish{\renewcommand{\figurename}{Figura}}
\addto\captionsspanish{\renewcommand{\tablename}{Tabla}}
%\usepackage{helvet}
%\renewcommand{\familydefault}{\sfdefault}
%% Asigna un tamaño a la hoja y los márgenes
\usepackage[a4paper,top=1cm,bottom=2cm,left=2cm,right=2cm,marginparwidth=2cm]{geometry}


%% Paquetes de la AMS
\usepackage{amsmath, amsthm, amsfonts}
%% Para añadir archivos con extensión pdf, jpg, png or tif
\usepackage{graphicx}
\usepackage[colorinlistoftodos]{todonotes}
\usepackage[colorlinks=true, allcolors=blue]{hyperref}
\setlength{\parindent}{0cm} 



\title{\vspace{0cm}
\textbf{LABORATORIO 1: }
\textbf{AJUSTE DE CURVAS}\\
\vspace{2mm}
}

\author{ 
\begin{tabular}{l  l  l  l }\\
{\bf Integrantes:}&{\bf Programa:}&{\bf Grupo:}&{\bf NPG:}\vspace{0.2cm}\\

{{- nom -}} & {{- Programa -}} & {{- grupo -}}& {{- nota -}}\\

\end{tabular}
\vspace{1cm}\\

\text{Fecha: } {{- calendario -}}
}

\date{}

\begin{document}

\maketitle
%
\vspace{2mm}
\section{\uppercase{Metodología}}
{{- metodologia -}}

\section{\uppercase {Resultados}}
{{- resultados -}}

\begin{figure}[H]
%\centering
\begin{tabular}{p{8.795cm}p{8.795cm}}
\begin{minipage}[l]{8.795cm}
\includegraphics[width=8.795cm]{ {{- grafplot -}} }
\caption{ {{- descripcion_grafica -}} }
\label{fig1}
\end{minipage}
&
\begin{minipage}[r]{7.45cm}
\begin{table}[H]
\begin{center}
\caption{ {{- descripcion_tabla -}} }
\vspace{0.5cm}
\label{tabla1}
\begin{tabular}{l l l l}
\hline
\vspace{0.001cm}
{{- name_ejes[0] -}}
&
{{- name_ejes[1] -}}
&
$\Delta${{- name_ejes[0] -}}
&
$\Delta${{- name_ejes[1] -}}\\
\hline

{{- mxdat[i,0] -}} &
{{- mxdat[i,1] -}} &
{{- mxdat[i,2] -}} &
{{- mxdat[i,3] -}}\\

\hline
\end{tabular}
\end{center}
\end{table}

\end{minipage}
\end{tabular}
\end{figure}


\section{\uppercase {Conclusiones}}
{{- conclusiones -}}

\section{\uppercase {Referencias}}

\end{document}

